\section{Introduction}

\subsection{Description of Investigation}
This investigation explores the effects of the tail plane presence and pitch angle on the longitudinal static stability, illustrated in figure 1. During the investigation the Lift, Drag, and Pitching moment acting on the aircraft model were measured in various configurations (no tail plane, tail plane angle of $+1^o$, and tail plane angle of $-3^o$). Wind tunnel corrections were applied to the results, and further corrections were made to account for the supports. Utilising various plots allowed for the formulation of various longitudinal stability constants of the aircraft given in the objectives. The aircraft is a twin engine business jet with an all moving tail plane \footnote{see Appendix A.1 for a diagram}. 
\begin{center}  
  \begin{figure}[h]
    \centering 
    % \includegraphics[scale=0.5]{LSS.PNG}
    \caption{Variation in flight Path of an aircraft after an initial disturbance depending on Longitudinal Static Stability.}
  \end{figure}
\end{center}

\subsection{Motivation}
Aircraft Stability is of great importance to Aeronautical engineers. Calculations regarding the change in stability with tail plane variation is crucial in understanding the behaviour of an aircraft. Stability and manoeuvrability of an aircraft is a trade off. Given that the aircraft being investigated is a passenger aircraft, one would expect that stability has a higher priority than manoeuvrability, especially in the longitudinal axis. 
\vspace{0.2cm}
\newline 
On the Japan Airways 123 flight part of the tail plane was damaged \cite{1}. This eventually lead to the aircraft Dutch rolling and experiencing 'unusual phugoid motion'. Both phenomena are due to a lack of stability of the aircraft - caused by the loss of the tail plane. The aircraft eventually crashed. This stresses the importance of aircraft stability. 

\subsection{Objectives}
\begin{enumerate}
  \item Measure Aerodynamic forces  and moments acting on the aircraft model, along with their various coefficients
  \item Determine  $K_n$, $x_{np}$, $\bar{x}_w$, $a_{w}$,  $a_{h}$, $\frac{d\epsilon}{d\alpha}$, $k$ , at each configuration \footnote {a dedicated list of Figures can be found in Appendix A.2}.
  \item Calculate the drag coefficients for each configuration.
 
\end{enumerate}