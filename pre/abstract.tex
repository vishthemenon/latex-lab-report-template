\begin{abstract}
  This report is an in depth description of an investigation into the stability of an aircraft tested in a wind tunnel. A twin engine passenger aircraft was tested in three configurations; tail off, tail set at $+1^o$, and tail set at $-3^o$. During the investigation various stability constants were calculated including the static stability margin, and the aerodynamic centre, through the use of various plots. This was achieved by measuring the lift, drag, and moment acting upon the aircraft at each incidence, varying from $-2^o$ up to roughly $10^o$. This report also details the corrections made to the values to account for the wind tunnel infrastructure. The stall behaviour was also observed to see how this impacted the drag acting on the aircraft, and how it affected the stability. The zero lift drag coefficient and the induced drag factor were also compared for each configuration.  
\end{abstract}
